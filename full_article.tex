\documentclass{proposalnsf}
%\usepackage{epsfig}

\usepackage{graphicx}
\usepackage{latexsym}
\usepackage{amsfonts,amsmath,amssymb}
\usepackage{url}
\usepackage[utf8]{inputenc}
\usepackage{fancyref}
\usepackage{hyperref}
\hypersetup{colorlinks=false,pdfborder={0 0 0},}

% NSF proposal generation template style file.
% based on latex stylefiles  written by Stefan Llewellyn Smith and
% Sarah Gille, with contributions from other collaborators.

\newcommand{\jas}{{\it J. Atmos. Sci.}}
\newcommand{\jpo}{{\it J. Phys. Oceanogr.}}
\newcommand{\JPO}{{\it J. Phys. Oceanogr.}}
\newcommand{\jfm}{{\it J. Fluid Mech.}}
\newcommand{\jgr}{{\it J. Geophys. Res.}}
\newcommand{\JGR}{{\it J. Geophys. Res.}}
\newcommand{\jmr}{{\it J. Mar. Res.}}
\newcommand{\arfm}{{\it Ann. Rev. Fluid Mech.}}
\newcommand{\dsr}{{\it Deep-Sea Res.}}
\newcommand{\dao}{{\it Dyn. Atmos. Oceans}}
\newcommand{\jam}{{\it Journal of Applied Meteorology}}
\newcommand{\phfl}{{\it Phys. Fluids}}
\newcommand{\phfla}{{\it Phys. Fluids A}}
\newcommand{\PhilTrans}{{\it Philosophical Transactions of the Royal Society, 
London}}
\newcommand{\gafd}{{\it Geophys. Astrophys. Fluid Dyn.}}
\newcommand{\gfd}{{\it Geophys. Fluid Dyn.}}
\newcommand{\PCE}   {{\it Physics and Chemistry of the Earth}}
\newcommand{\PRL}   {{\it Physical Review Letters}}

\newcommand{\ProgOc}{{\it Prog. Oceanography}}
\newcommand{\WHOITR}{Woods Hole Oceanographic Institution Technical Report, WHOI-}
\newcommand{\degrees}{$\!\!$\char23$\!$}
\DeclareFontFamily{OT1}{psyr}{}
\DeclareFontShape{OT1}{psyr}{m}{n}{<-> psyr}{}
\def\times{{\fontfamily{psyr}\selectfont\char180}}

\newcommand{\truncateit}[1]{\truncate{0.8\textwidth}{#1}}
\newcommand{\scititle}[1]{\title[\truncateit{#1}]{#1}}


\renewcommand{\refname}{\centerline{References cited}}

% this handles hanging indents for publications
\def\rrr#1\\{\par
\medskip\hbox{\vbox{\parindent=2em\hsize=6.12in
\hangindent=4em\hangafter=1#1}}}

\def\baselinestretch{1}

\begin{document}

\bibliographystyle{jponew}

\pagenumbering{arabic}
\renewcommand{\thepage} {\arabic{page}}

\noindent{\Large \bf PROJECT DESCRIPTION}




\section{Plans for data management and sharing of the products of research.}

Proposals must include a supplementary document of no more than
two pages labeled ``Data Management Plan''. This supplement should
describe:

\begin{list}
\item 1.~~the types of data, samples, physical collections, software, curriculum materials, and other materials to be produced in the course of the project;

\item 2.~~the standards to be used for data and metadata format and content (where existing standards are absent or deemed inadequate, this should be documented along with any proposed solutions or remedies);

\item 3.~~policies for access and sharing including provisions for appropriate protection of privacy, confidentiality, security, intellectual property, or other rights or requirements;

\item 4.~~policies and provisions for re-use, re-distribution, and the production of derivatives; and

\item 5.~~plans for archiving data, samples, and other research products, and for preservation of access to them.

\end{list}


%%%%%%%%% SUMMARY -- 1 page, third person
% e.g:  "The PI will prove" not "I will prove"

\section{Project Summary}
This should be a brief statement of the problem you plan to address.
It should look something like an abstract. 

\section{Intellectual Merit}
% This is why your project is interesting and will help further
% knowledge in the field of mathematics. 

\section{Broader Impacts}
There are 4 kinds of broader impacts.
1. advance discovery and understanding while promoting teaching,
training and learning
2. broaden the participation of underrepresented groups
3. disseminated broadly to enhance scientific and technological
understanding
4. benefits of the proposed activity to society



%%%%%%%%% PROPOSAL -- 15 pages (including Prior NSF Support)

\section{Project Description}

%From the NSF Grants Proposal Guide:
%"The Project Description should provide a clear statement of the work 
%to be undertaken and must include: objectives for the period of the proposed 
% work and expected significance; relation to longer-term goals of the PI's 
% project; and relation to the present state of knowledge in the field, 
% to work in progress by the PI under other support and to work in progress 
% elsewhere."

In my paper \cite{paper01} written during the period of my present
grant ... 


\section{Broader Impacts}
% as in the project summary, broader impacts must be called out separately 
% in the project description.  You may be able to give more specific
% examples, or discuss how you've previously achieved these impacts.
% It should be similar, but not identical, to the Broader Impacts statement
% in the project summary

\section{Results From Prior NSF Support}
% 5 pages or fewer of the 15 pages for entire description document.
% include results from NSF grants received in the past 5 years.
% if supported by more than one grant, choose the most relevant one
% for each grant, include: NSF award number, amount, dates of
% support, and publications resulting from this research.
% due to space limitations, it is often advisable to use citations rather
% than putting the titles of the publications in the body 
% of this section

% e.g.: "My prior grant, "Uses of Coffee in Mathematical Research" (DMS-0123456, 
% $100,000, 2005-2008), resulted in 3 papers [1],[2],[3], demonstrating..."

% if requesting postdoctoral research salary, a supplemental 1-page document
% called "Postdoc Mentoring Plan" will be required 


%%%%%%%%% BIOGRAPHICAL SKETCH -- 2 pages

\section{Biographical Sketch: Your Name}

Your Bio should be divided into the following sections
(a) Professional Preparation (education):
Undergrad, Major, Year
Graduate, Major, Year
Postdoc, Area, Years-Inclusive
(b) Appointments:  most recent first.
(c) Publications:  5 related to the proposal, and 5 "Other Significant Publications"
(d) Synergistic Activities (math-enhancing activities that were not
part of your main job description, like editorial boards and
conference organizing - any Math-related volunteer work.
these are often similar to Broader Impacts
(e) Collaborators & Other Affiliations: (use the following sections)
list in alphabetical order, and include current affiliations parenthetically
Collaborators and Co-editors: past 48 months.  If none, write "none"
Graduate Advisors and Postdoctoral sponsors: (your own, no matter how long ago)
Thesis advisor and postgraduate scholar-sponsor:  those you have advised
in the past 5 years.  
Total number of graduate students advised: X (all time)
Total number of postdoctoral scholars sponsored: Y (all time)

\bibliography{bibliography/converted_to_latex.bib}

\end{document}

